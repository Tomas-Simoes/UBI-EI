\section{Configurações de Roteamento}
\label{sec:confroute}
De modo a que as quatro "redes principais" da UBI, Gigapix Lisboa/Porto e Google
se comunicassem de forma eficiente e confiável tivemos de implementar o roteamento entre
essas redes através dos protocologos OSPF, RIP e Frame Relay.

\subsection{Protocolo OSPF}
Este é empregado pelo roteamento entre a rede da UBI e da rede da FCNN.

Foi definido da seguinte forma:

\vspace{3mm}
\setlength{\tabcolsep}{20pt}
\renewcommand{\arraystretch}{1.5}
\noindent
\begin{tabular}{ |l|l|l|}
  \hline
  \multicolumn{3}{|c|}{Roteamento OSPF}                               \\
  \hline
  Router                            & ID      & Rotas Conhecidas      \\
  \hline
  \multirow{2}{1em}{Gigapix Lisboa} & 1.1.1.1 & 193.136.66.0 (UBI)    \\
                                    &         & 193.136.1.0 (Porto)   \\
  \hline
  \multirow{2}{1em}{Gigapix Porto}  & 2.2.2.2 & 193.136.66.4 (UBI)    \\
                                    &         & 193.136.1.0 (Lisboa)  \\
  \hline
  \multirow{2}{1em}{UBI}            & 3.3.3.3 & 193.136.66.0 (Lisboa) \\
                                    &         & 193.136.66.4 (Porto)  \\
  \hline
\end{tabular}
\vspace{5mm}

\subsection{Protocolo Frame Relay}
Este tem como objetivo fazer a conexão entre os \textit{routers} da FCNN e da Google
Foi definido da seguinte forma:

\vspace{3mm}
\setlength{\tabcolsep}{20pt}
\renewcommand{\arraystretch}{1.5}
\noindent
\begin{tabular}{ |p{3cm}|p{3cm}|p{1cm}|}
  \hline
  \multicolumn{3}{|c|}{Roteamento do Frame-Relay} \\
  \hline
  Rota            & IP            & DLCI          \\
  \hline
  Lisboa - Google & 10.10.10.0/30 & 100/101       \\
  Porto - Google  & 10.10.20.0/30 & 200/201       \\
  Lisboa - Porto  & 10.10.30.0/30 & 300/301       \\

  \hline
\end{tabular}
\vspace{5mm}

\subsection{Protocolo RIP}
Este funciona como um protocolo auxiliar do OSPF, de modo a garantir a
comunicação entre os routers da Google, FCNN e UBI.
Foi definido da seguinte forma:

\vspace{3mm}
\setlength{\tabcolsep}{20pt}
\renewcommand{\arraystretch}{1.5}
\noindent
\begin{tabular}{ |p{2cm}|p{3.25cm}|}
  \hline
  \multicolumn{2}{|c|}{Roteamento RIP}                   \\
  \hline
  Router                            & Rotas Distribuidas \\
  \hline
  \multirow{3}{1em}{UBI}            & 10.0.0.0           \\
                                    & 193.136.66.0       \\
                                    & 193.136.67.0       \\
  \hline
  \multirow{2}{1em}{Gigapix Lisboa} & 10.0.0.0           \\
                                    & 193.136.66.0       \\
  \hline
  \multirow{2}{1em}{Google}         & 10.0.0.0           \\
                                    & 8.0.0.0            \\
  \hline
\end{tabular}
\vspace{5mm}