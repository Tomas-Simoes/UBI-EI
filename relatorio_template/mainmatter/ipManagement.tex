\section{Distribuição de IP's}
\label{sec:distip}
Este capítulo aborda o processo de planeamento e implementação da distribuição
de IPs e máscaras na nossa infraestrutura. O objetivo é alcançar uma configuração eficiente e escalável,
garantindo um desempenho otimizado.

\subsection{Rede da UBI}
Na rede da UBI a informação que tinhamos é que tinham de ser \textit{subnets} da rede \textit{10.0.0.0/16}.

Devido ao tamanho da UBI e a quantidade de \textit{VLANs} que esta tem, tivemos de decidir qual
seria a melhor máscara de modo a que não falta-se \textit{hosts} para cada departamento.

Fizemos então a gestão na rede da UBI da seguinte forma:

\vspace{3mm}
\setlength{\tabcolsep}{20pt}
\renewcommand{\arraystretch}{1.5}
\noindent
\begin{tabular}{ |p{3cm}|p{3cm}|p{5cm}|}
  \hline
  \multicolumn{3}{|c|}{Rede da UBI}                                         \\
  \hline
  VLAN            & IP               & IP Range                             \\
  \hline
  DI-Docentes     & 10.0.0.0/20      & 10.0.0.1 to 10.0.15.254              \\
  DI-Geral        & 10.0.16.0/20     & 10.0.16.1 to 10.0.31.254             \\
  D-Matemática    & 10.0.32.0/20     & 10.0.32.1 to 10.0.47.254             \\
  ...             & ...              & ...                                  \\
  EDURAM          & 10.0.192.0/20    & 10.0.224.1 to 10.0.239.254           \\
  Private Servers & 10.0.240.0/20    & 10.0.240.1 to 10.0.255.254           \\
  Public Servers  & 193.136.67.0/255 & \small193.136.67.1 to 193.136.67.254 \\
  \hline
\end{tabular}
\vspace{2mm}

Com as VLANs repartidas com uma máscara de rede /20 permite-nos ter 16 \textit{subnets}, com 4096 \textit{hosts} cada. \\

\break
Para os servidores da UBI, decidimos um IP dentro de cada uma da sua \textit{subnet}.

\vspace{3mm}
\setlength{\tabcolsep}{20pt}
\renewcommand{\arraystretch}{1.5}
\noindent
\begin{tabular}{ |p{3cm}|p{3cm}|p{5cm}|}
  \hline
  \multicolumn{3}{|c|}{Servidores da UBI}      \\
  \hline
  Servidor    & IP             & Mask          \\
  \hline
  www.ubi.pt  & 193.163.67.208 & 255.255.255.0 \\
  mail.ubi.pt & 193.163.67.208 & 255.255.255.0 \\
  DNS UBI     & 10.0.240.2     & 255.255.240.0 \\
  \hline
\end{tabular}
\vspace{5mm}

\subsection{Rede da FCCN}
Na rede da FCCN a informação que tinhamos é que tinham de ser \textit{subnets} da rede
\textit{193.136.0.0/16}.

Novamente para pouparmos o máximo de \textit{IP's} possiveis decidimos usar
uma máscara de \textit{255.255.255.252} entre os \textit{routers} da
Gigapix Lisboa, Gigapix Porto e UBI. Dessa forma em cada ligação só gastavamos 4 \textit{IP's}.

Fizemos então a gestão da rede da seguinte forma:

\vspace{3mm}
\setlength{\tabcolsep}{20pt}
\renewcommand{\arraystretch}{1.5}
\noindent
\begin{tabular}{ |p{3cm}|p{3cm}|p{1cm}|}
  \hline
  \multicolumn{3}{|c|}{Ligações do FCCN}   \\
  \hline
  Rota           & IP              & Hosts \\
  \hline
  UBI - Lisboa   & 193.163.66.0/30 & 2     \\
  UBI - Porto    & 193.163.66.4/30 & 2     \\
  Lisboa - Porto & 193.163.1.0/30  & 2     \\

  \hline
\end{tabular}
\vspace{5mm}

\break
\subsection{Restantes Redes}
Ainda precisámos de configurar a rede da Google e do ISPx. Para isso utilizamos os \textit{IP's} que foram fornecidos no enunciado.

Para a rede do ISPx, foi usado o \textit{192.168.1.0/24}.

Já para a Google, cada servidor tem a sua \textit{subnet} do \textit{8.8.0.0/16}

\vspace{3mm}
\setlength{\tabcolsep}{20pt}
\renewcommand{\arraystretch}{1.5}
\noindent
\begin{tabular}{ |p{3cm}|p{2cm}|p{2cm}|}
  \hline
  \multicolumn{3}{|c|}{Servidores da Google}   \\
  \hline
  Servidor       & IP          & Mask          \\
  \hline
  www.google.com & 8.8.0.2     & 255.255.255.0 \\
  gmail.com      & 8.8.200.111 & 255.255.255.0 \\
  DNS Google     & 8.8.8.8     & 255.255.240.0 \\
  \hline
\end{tabular}
\vspace{5mm}

