\section{Introdução}
\label{chap:intro}
Este relatório, com o tema de "Governance da Internet" foi feito no contexto da unidade curricular de Redes de Computador, no projeto
enquanto frequentava a \ac{UBI}.

Atualmente vivemos em uma sociedade que é governada pela tecnologica, apesar de a maioria dos individuos
não saberem quem a governa.

Ao longo dos anos tivemos diferentes perpetivas sobre o uso desta,
desde o uso militar até simplemente podermos usar o Youtube num frigorifico. Por essa razão
é necessário atualizar e dialogar sobre como deviamos nos comportar face a questões políticas, sociais e filosoficas da rede de redes
de modo a termos um espaço seguro para estar.

Este documento fala sobre o aparecimento do termo "Governancia da Internet", um campo de estudo e pesquisa,
complexo e com diversas intrepetações sobre quem  dentro da Internet.

\subsection{Organização do Documento}
\label{sec:organ}

De modo a refletir o trabalho que foi feito, este documento encontra-se estruturado da seguinte forma:
\begin{enumerate}
  \item O 1º capítulo -- \textbf{Introdução} -- apresenta o projeto, a motivação para a sua escolha, o enquadramento para o mesmo, os seus objetivos e a respetiva organização do documento.
  \item O 2º capítulo -- \textbf{Tecnologias Utilizadas} -- descreve os conceitos mais importantes no âmbito deste projeto, bem como as tecnologias utilizadas durante do desenvolvimento da aplicação.
  \item ...
\end{enumerate}
