\section{Introdução}
\label{sec:intro}
Este relatório, com o tema de "Governancia da Internet" foi feito no contexto da unidade curricular de Redes de Computador, enquanto frequentava a Universidade da Beira Interior.

Ao longo dos anos tivemos diferentes perpetivas sobre o uso desta,
desde o uso militar até simplemente podermos usar o Youtube num frigorifico. Por essa razão
é necessário atualizar e dialogar sobre como deviamos nos comportar face a questões políticas, sociais e filosoficas da rede de redes
de modo a termos um espaço seguro para estar.

Este documento fala sobre o aparecimento do termo "Governancia da Internet", um campo de estudo e pesquisa,
complexo e com diversas intrepetações sobre quem governa a Internet numa socieadade governada por ela.

\subsection{Organização do Documento}

De modo a refletir o trabalho que foi feito, este documento encontra-se estruturado da seguinte forma:
\begin{enumerate}
  \item O capítulo "\nameref{sec:intro} apresenta o relatório, os seus objetivos e a respetiva organização do documento.
  \item O capítulo "\nameref{sec:internetgovernance}" descreve os conceitos mais importantes no âmbito deste projeto, bem como as tecnologias utilizadas durante do desenvolvimento da aplicação.
  \item O capítulo "\nameref{sec:beggining}"
  \item O capítulo "\nameref{sec:governanceevo}" descreve os conceitos mais importantes no âmbito deste projeto, bem como as tecnologias utilizadas durante do desenvolvimento da aplicação.
\end{enumerate}
