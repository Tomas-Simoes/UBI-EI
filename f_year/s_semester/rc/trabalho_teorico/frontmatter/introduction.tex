\section{Introdução}
\label{sec:intro}
Ao longo dos anos tivemos diferentes perpetivas sobre o uso desta,
desde o uso militar até simplemente podermos usar o Youtube num frigorifico. Por essa razão
é necessário atualizar e dialogar sobre como deviamos nos comportar face a questões políticas, sociais e filosoficas 
dentro da Rede de Redes, de modo a termos um espaço seguro para estar.

Este documento, com o tema de "Governança e Evolução da Internet" no contexto da unidade 
curricular de Redes de Computadores, fala sobre o aparecimento do termo 
"Governança da Internet", um campo de estudo e pesquisa, complexo e com diversas 
intrepetações sobre como deviamos lidar com essas questões, assim como a evolução da mesma.

\subsection{Organização do Documento}

De modo a refletir o trabalho que foi feito, este documento encontra-se estruturado da seguinte forma:
\begin{enumerate}
  \item O capítulo "\nameref{sec:intro}" apresenta o relatório, os seus objetivos e a respetiva organização do documento;
  
  \item O capítulo "\nameref{sec:internetgovernance}" explora o conceito de Governança da Internet; 
  
\item O capítulo "\nameref{sec:governanceevo}" aborda a evolução desta, desde suas origens até o desenvolvimento da ICANN e a expansão para além da 
coordenação global de nomes de domínio e IPs.;
  
\item O capítulo "\nameref{sec:organization}" fornece uma visão geral da organização da 
Internet, incluindo sua infraestrutura física, o papel dos provedores de serviços de Internet 
e os protocolos de comunicação fundamentais. 
  
\item O capítulo "\nameref{sec:portugalcase}" examina a história e o 
desenvolvimento da Internet em Portugal, desde seus primeiros passos até a 
modernização e democratização do acesso. 
\end{enumerate}
