\documentclass[12pt,a4paper]{article}

% PACOTES
\usepackage[portuguese]{babel}
\usepackage[utf8]{inputenc}
\usepackage[T1]{fontenc}
\usepackage{makeidx}
\usepackage{xspace}
\usepackage{graphicx,color,times}
\usepackage{fancyhdr}
\usepackage{latexsym}
\usepackage[printonlyused]{acronym} % Pacote acronym aqui
\usepackage{float}
\usepackage{listings}
% remove duplicate "index"
\usepackage[nottoc]{tocbibind}
\usepackage{natbib}
\usepackage{hyperref}
\usepackage{xcolor}
\usepackage{fix-cm}
\usepackage{fourier}
\usepackage[scaled=.92]{helvet}
% show references names
\usepackage{nameref}

% CONFIGURAÇÕES HYPERREF
\hypersetup{
    colorlinks=true,            
    linkcolor=blue,            
    citecolor=blue,             
    urlcolor=blue,              
    filecolor=blue,             
    breaklinks=true,            
    pdftitle={A Governancia da Internet},   
    pdfauthor={Tomás Simões Duarte},           
    pdfsubject=
    {
        As decisões no design e desenvolvimento 
        de tecnologias que são necessárias para manter a Internet operacional 
        com a implementação
        de políticas novas em volta das mesmas
    }, 
    pdfkeywords={Governancia, Internet},   
}

% DEFINIÇÃO DE ACRÔNIMOS
\begin{document}

% Frontmatter
\thispagestyle{empty}
\setcounter{page}{-1}

\begin{center}
  \begin{Huge}
    \textbf{Universidade da Beira Interior}
  \end{Huge}
\end{center}

\begin{center}
  \begin{Huge}
    Departamento de Informática
  \end{Huge}
\end{center}

\vspace{0,07cm}
\begin{figure}[!htb]
  \centering
  \includegraphics[width=191pt]{images/ubi-fe-di.png}
\end{figure}

\vspace{0.5cm}
\begin{center}
  \begin{Large}
    \textbf{QUEM GOVERNA A INTERNET}
  \end{Large}
\end{center}


\vspace{0.5cm}
\begin{center}
  \begin{normalsize}
    \begin{large}
      Elaborado por:
    \end{large}
  \end{normalsize}
\end{center}

\vspace{0.2cm}
\begin{center}
  \begin{large}
    \textbf{Tomás Simões [52585]}
  \end{large}
\end{center}

\vspace{0,5cm}
\begin{center}
  \begin{normalsize}
    \begin{large}
      Orientador:
    \end{large}
  \end{normalsize}
\end{center}

\vspace{0.2cm}
\begin{center}
  \begin{large}
    \textbf{Professor Doutor Bruno Miguel Silva}
  \end{large}
\end{center}



\vspace{0.5cm}
\begin{center}
  \begin{normalsize}
    \today
  \end{normalsize}
\end{center}

\let\cleardoublepage\clearpage
\tableofcontents

% Mainmatter
\acresetall
\section{Introdução}
\label{sec:intro}
Este documento, realizado no âmbito da unidade curricular de Redes de Computadores, tem como objetivo
divulgar a nossa abordagem face a estrutura da rede da Universidade da Beira Interior em termos de distribuição de IP's e
configuração dos dispositivos.

\subsection{Organização do Documento}

De modo a refletir o trabalho que foi feito, este documento encontra-se estruturado da seguinte forma:
\begin{enumerate}
  \item O capítulo "\nameref{sec:intro}" apresenta o relatório, os seus objetivos e a respetiva organização do documento;

  \item O capítulo "\nameref{sec:distip}" se concentra em detalhar o processo de planejamento e implementação da
        distribuição de IPs e máscaras

  \item O capítulo "\nameref{sec:confroute}" aborda a configuração detalhada dos protocolos de roteamento OSPF, RIP e Frame Relay
        para interligar a rede da Universidade da Beira Interior (UBI), da FCNN e do Google.;

\end{enumerate}

\section{O que é a Governancia da Internet?}
\label{sec:internetgovernance}
A Governancia da Internet é definida como "decisões no \textit{design} e desenvolvimento 
de tecnologias que são necessárias para manter a Internet operacional com a implementação
de políticas novas em volta das mesmas".

Ao longo do tempo, pesquisadores assim como profissionais na área têm tentado expandir o 
conhecimento deste extenso dominio, apresentando divisões para os problemas em questão.

Uma dessas classificações, proposta por DeNardis e Raymond, segmenta a Governança da Internet em seis áreas funcionais:

\begin{enumerate}
  \item Controle de Recursos Críticos da Internet (CIRs);
  \item Estabelecimento de \textit{standards} da Internet;
  \item Coordenação de Acesso e Interligação;
  \item Governança de Cibersegurança;
  \item Intermediação de Informações;
  \item Direitos de Propriedade Intelectual baseados em Arquitetura;
\end{enumerate}

Estas áreas têm diferentes estratégias e modelos de admistração, cada um adaptado
para ser mais eficiente na sua área especifica. No entanto, todos os pontos 
convergem para um objetivo comum.

Examinando a trajetória da Internet ao longo do tempo, percebe-se que ela foi 
observada em diferentes contextos históricos e culturais, acompanhada de 
diversas expectativas e decepções.

Por conta disso, foi preciso mantar uma abordagem de flexivel face ao desenvolvimento
da arquitetura da Internet, permitindo que novos atores com diferentes ideias entrassem 
e modificassem a estrutura conforme desejado.





\section{O inicio da Internet}
\label{sec:beggining}

\subsection{A bomba atómica e a Internet}
O primeiro passo na criação da Rede de Redes foi feito em 29 de agosto de 1949
quando a União Soviética detonou a sua primeira arma nuclear e os Estados Unidos
começaram a trabalhar nas suas própias técnologias, como um estação de radares
contra força-area. 

No meio disso, varios acontecimentos da Guerra Fria aconteceram, que não vamos
entrar a fundo. No entanto, tudo fez com que alguns anos depois
a ARPANET surgiu como uma defesa aos ataques da União Soviética.

O \textit{US Department of Defense} (DoD) estava preocupada em como 
os seus lideres politicos e militares iam continuar conectados após
um ataque nuclear inicial por parte da União Soviética.

A solução técnica foi criar novas maneiras de \textit{routing} e \textit{switching}
de forma a terem sistemas de comunicação decentrealizados que poderiam sobreviver 
a qualquer ataque.

O sistema proposto se assemelhava notavelmente ao que a ARPANET eventualmente
se tornou.

Podemos assim dizer que a criação Internet teve uma relação bastante
intima com a criação da bomba atómica.

\subsection{Surgimento da ARPANET}
Criada pelo \textit{US Defense Advanced Reasearch Projects Agency} (ARPA), foi a primeira
rede de computadores experimental estabelicida, onde novos \textit{softwares} 
eram testados. Esta rede ligava 4 universidades, o que também permitia 
a divulgação de informação de forma muito mais eficaz.

Esta funcionava através de um sistema de transmissão de dados no qual as informações
são divididas em pequenos pacotes, que por sua vez contém trechos dos dados e
o endereço do destinatário. Trouxe outras estratégias inovadoras como 
o "\textit{layering}", que permitia que os componenetes da rede podessem
ser alterados de forma independente.

A ideia da rede com "\textit{open-architecture}" surgiu logo em 1972, por Kahn que 
definiu quatro regras:

\begin{enumerate}
  \item Cada rede deveria ser capaz de funcionar independentemente e não precisaria 
passar por mudanças internas para se conectar à Internet.
  \item A transmissão de dados seria feita com a melhor fidelidade possível. 
Se um pacote de dados não chegasse ao destino, ele seria rapidamente retransmitido a 
partir da origem.
  \item Dispositivos conhecidos hoje em dia como \textit{gateways} e \textit{rooters} seriam \
utilizados para conectar as redes. Esses dispositivos não manteriam informações 
sobre os pacotes que passam por eles, mantendo assim uma  estrutura simples e evitando adaptações complicadas.
\end{enumerate}

Durante a década de 70 também surgiram novos protocologos de comunicação, incluido o TCP/IP, que se
tornou o padrão para a ARPANET e as suas redes.

Ao longo do tempo, varias versões da ARPANET surgiram e a humanidade continou a avançar 
na ideia central de uma rede de comunicação.




\section{Evolução da Governancia da Internet}
\label{sec:governanceevo}
O termo "Governancia da Internet" não está em nenhum artigo escolar até 1995. No entanto começaram a surgir pesquisadores com propostas em como aplicar 
governancia e leis na Internet.

O surgimento desta área de pesquisa converge para acontecimentos em meados do século 19, 
em que a Internet começou a emergir como uma massiva mídia
causada por 3 principais eventos:

\begin{enumerate}
  \item O desenvolvimento e adoção do protocologo WWW (1989-1993);
  \item[]Permitiu a criação de uma rede de informações interligadas que poderiam
  ser acessadas por meio de navegadores da \textit{web}.
  
  \item A publicação de \textit{web browsers} gratuitos para utilização (1991);
  
  \item[]Os \textit{web browsers} geralmente eram comercializados e a disponibilidade dos mesmos 
  facilitou o acesso à Internet e abriu portas para novos individuos na sua população.
  \item Privatização da Internet e a sua abertura para uso comercial pela \textit{US National Science Foundation} (1995) 

  \item[]Antigamente, a infraestrutura principal da Internet era predominanmente controlada e financiada pelo
  governo ou instituições acadêmicas, mas a decisão de permitir o uso comercial abriu caminho
  para o desenvolvimento do comércio eletrónico.


\end{enumerate}

% \include{chap-5}
% \include{chap-6}
% \chapter*{Conclusão}
\addcontentsline{toc}{chapter}{Conclusão}
\label{chap:conc-trab-futuro}

Esta secção contém a resposta à questão: \\
\emph{Quais foram as conclusões princípais a que o(a) aluno(a) chegou no fim deste trabalho?}

Esta secção responde a questões como:\\
\emph{O que é que ficou por fazer, e porque?}\\
\emph{O que é que seria interessante fazer, mas não foi feito por não ser exatamente o objetivo deste trabalho?}\\
\emph{Em que outros casos ou situações ou cenários -- que não foram estudados no contexto deste projeto por não ser seu objetivo -- é que o trabalho aqui descrito pode ter aplicações interessantes e porque?}
teste 1 2 \cite{musiani2013network}.
\bibliography{references}
\end{document}