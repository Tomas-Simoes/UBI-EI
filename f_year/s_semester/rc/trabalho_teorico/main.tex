\documentclass[12pt,a4paper]{article}

% PACOTES
\usepackage[portuguese]{babel}
\usepackage[utf8]{inputenc}
\usepackage[T1]{fontenc}
\usepackage{makeidx}
\usepackage{xspace}
\usepackage{graphicx,color,times}
\usepackage{fancyhdr}
\usepackage{latexsym}
\usepackage[printonlyused]{acronym} % Pacote acronym aqui
\usepackage{float}
\usepackage{listings}
% remove duplicate "index"
\usepackage[nottoc]{tocbibind}
\usepackage{natbib}
\usepackage{hyperref}
\usepackage{xcolor}
\usepackage{fix-cm}
\usepackage{fourier}
\usepackage[scaled=.92]{helvet}
% show references names
\usepackage{nameref}

% CONFIGURAÇÕES HYPERREF
\hypersetup{
    colorlinks=true,            
    linkcolor=black,            
    citecolor=blue,             
    urlcolor=blue,              
    filecolor=blue,             
    breaklinks=true,            
    pdftitle={A Governança da Internet},   
    pdfauthor={Tomás Simões Duarte},           
    pdfsubject=
    {
        As decisões no design e desenvolvimento 
        de tecnologias que são necessárias para manter a Internet operacional 
        com a implementação
        de políticas novas em volta das mesmas
    }, 
    pdfkeywords={Governança, Internet},   
}

% DEFINIÇÃO DE ACRÔNIMOS
\begin{document}

% Frontmatter
\thispagestyle{empty}
\setcounter{page}{-1}

\begin{center}
  \begin{Huge}
    \textbf{Universidade da Beira Interior}
  \end{Huge}
\end{center}

\begin{center}
  \begin{Huge}
    Departamento de Informática
  \end{Huge}
\end{center}

\vspace{0,07cm}
\begin{figure}[!htb]
  \centering
  \includegraphics[width=191pt]{images/ubi-fe-di.png}
\end{figure}

\vspace{0.5cm}
\begin{center}
  \begin{Large}
    \textbf{QUEM GOVERNA A INTERNET}
  \end{Large}
\end{center}


\vspace{0.5cm}
\begin{center}
  \begin{normalsize}
    \begin{large}
      Elaborado por:
    \end{large}
  \end{normalsize}
\end{center}

\vspace{0.2cm}
\begin{center}
  \begin{large}
    \textbf{Tomás Simões [52585]}
  \end{large}
\end{center}

\vspace{0,5cm}
\begin{center}
  \begin{normalsize}
    \begin{large}
      Orientador:
    \end{large}
  \end{normalsize}
\end{center}

\vspace{0.2cm}
\begin{center}
  \begin{large}
    \textbf{Professor Doutor Bruno Miguel Silva}
  \end{large}
\end{center}



\vspace{0.5cm}
\begin{center}
  \begin{normalsize}
    \today
  \end{normalsize}
\end{center}

\let\cleardoublepage\clearpage
\tableofcontents

% Mainmatter
\acresetall
\section{Introdução}
\label{sec:intro}
Este documento, realizado no âmbito da unidade curricular de Redes de Computadores, tem como objetivo
divulgar a nossa abordagem face a estrutura da rede da Universidade da Beira Interior em termos de distribuição de IP's e
configuração dos dispositivos.

\subsection{Organização do Documento}

De modo a refletir o trabalho que foi feito, este documento encontra-se estruturado da seguinte forma:
\begin{enumerate}
  \item O capítulo "\nameref{sec:intro}" apresenta o relatório, os seus objetivos e a respetiva organização do documento;

  \item O capítulo "\nameref{sec:distip}" se concentra em detalhar o processo de planejamento e implementação da
        distribuição de IPs e máscaras

  \item O capítulo "\nameref{sec:confroute}" aborda a configuração detalhada dos protocolos de roteamento OSPF, RIP e Frame Relay
        para interligar a rede da Universidade da Beira Interior (UBI), da FCNN e do Google.;

\end{enumerate}

\section{O que é a Governancia da Internet?}
\label{sec:internetgovernance}
A Governancia da Internet é definida como "decisões no \textit{design} e desenvolvimento 
de tecnologias que são necessárias para manter a Internet operacional com a implementação
de políticas novas em volta das mesmas".

Ao longo do tempo, pesquisadores assim como profissionais na área têm tentado expandir o 
conhecimento deste extenso dominio, apresentando divisões para os problemas em questão.

Uma dessas classificações, proposta por DeNardis e Raymond, segmenta a Governança da Internet em seis áreas funcionais:

\begin{enumerate}
  \item Controle de Recursos Críticos da Internet (CIRs);
  \item Estabelecimento de \textit{standards} da Internet;
  \item Coordenação de Acesso e Interligação;
  \item Governança de Cibersegurança;
  \item Intermediação de Informações;
  \item Direitos de Propriedade Intelectual baseados em Arquitetura;
\end{enumerate}

Estas áreas têm diferentes estratégias e modelos de admistração, cada um adaptado
para ser mais eficiente na sua área especifica. No entanto, todos os pontos 
convergem para um objetivo comum.

Examinando a trajetória da Internet ao longo do tempo, percebe-se que ela foi 
observada em diferentes contextos históricos e culturais, acompanhada de 
diversas expectativas e decepções.

Por conta disso, foi preciso mantar uma abordagem de flexivel face ao desenvolvimento
da arquitetura da Internet, permitindo que novos atores com diferentes ideias entrassem 
e modificassem a estrutura conforme desejado.





\section{O inicio da Internet}
\label{sec:beggining}

\subsection{A bomba atómica e a Internet}
O primeiro passo na criação da Rede de Redes foi feito em 29 de agosto de 1949
quando a União Soviética detonou a sua primeira arma nuclear e os Estados Unidos
começaram a trabalhar nas suas própias técnologias, como um estação de radares
contra força-area. 

No meio disso, varios acontecimentos da Guerra Fria aconteceram, que não vamos
entrar a fundo. No entanto, tudo fez com que alguns anos depois
a ARPANET surgiu como uma defesa aos ataques da União Soviética.

O \textit{US Department of Defense} (DoD) estava preocupada em como 
os seus lideres politicos e militares iam continuar conectados após
um ataque nuclear inicial por parte da União Soviética.

A solução técnica foi criar novas maneiras de \textit{routing} e \textit{switching}
de forma a terem sistemas de comunicação decentrealizados que poderiam sobreviver 
a qualquer ataque.

O sistema proposto se assemelhava notavelmente ao que a ARPANET eventualmente
se tornou.

Podemos assim dizer que a criação Internet teve uma relação bastante
intima com a criação da bomba atómica.

\subsection{Surgimento da ARPANET}
Criada pelo \textit{US Defense Advanced Reasearch Projects Agency} (ARPA), foi a primeira
rede de computadores experimental estabelicida, onde novos \textit{softwares} 
eram testados. Esta rede ligava 4 universidades, o que também permitia 
a divulgação de informação de forma muito mais eficaz.

Esta funcionava através de um sistema de transmissão de dados no qual as informações
são divididas em pequenos pacotes, que por sua vez contém trechos dos dados e
o endereço do destinatário. Trouxe outras estratégias inovadoras como 
o "\textit{layering}", que permitia que os componenetes da rede podessem
ser alterados de forma independente.

A ideia da rede com "\textit{open-architecture}" surgiu logo em 1972, por Kahn que 
definiu quatro regras:

\begin{enumerate}
  \item Cada rede deveria ser capaz de funcionar independentemente e não precisaria 
passar por mudanças internas para se conectar à Internet.
  \item A transmissão de dados seria feita com a melhor fidelidade possível. 
Se um pacote de dados não chegasse ao destino, ele seria rapidamente retransmitido a 
partir da origem.
  \item Dispositivos conhecidos hoje em dia como \textit{gateways} e \textit{rooters} seriam \
utilizados para conectar as redes. Esses dispositivos não manteriam informações 
sobre os pacotes que passam por eles, mantendo assim uma  estrutura simples e evitando adaptações complicadas.
\end{enumerate}

Durante a década de 70 também surgiram novos protocologos de comunicação, incluido o TCP/IP, que se
tornou o padrão para a ARPANET e as suas redes.

Ao longo do tempo, varias versões da ARPANET surgiram e a humanidade continou a avançar 
na ideia central de uma rede de comunicação.




\section{Evolução da Governança da Internet}
\label{sec:governanceevo}
O termo "Governança da Internet" não está em nenhum artigo escolar até 1995. No entanto começaram a surgir pesquisadores com propostas em como aplicar 
Governança e leis na Internet.

O surgimento desta área de pesquisa converge para acontecimentos em meados do século 19, 
em que a Internet começou a emergir como uma massiva mídia
causada por 3 principais eventos:

\begin{enumerate}
  \item O desenvolvimento e adoção do protocologo WWW (1989-1993);
  \item[] Introduzindo a ideia de páginas web acessíveis a todos e permitiu a criação 
de uma rede de informações interligadas que poderiam ser acessadas por meio de navegadores
da \textit{web}.
  
  \item O surgimento de provedores de serviços de Internet (ISPs);
  \item[] Levou à popularização do acesso à Internet em domicilios e empresas;

  \item A publicação de \textit{web browsers} gratuitos para utilização (1991);
  \item[]Os \textit{web browsers} geralmente eram comercializados e a disponibilidade 
dos mesmos facilitou o acesso à Internet e abriu portas para novos individuos na sua
população.

  \item Privatização da Internet e a sua abertura para uso comercial 
pela \textit{US National Science Foundation} (1995); 
  \item[]Antigamente, a infraestrutura principal da Internet era predominanmente 
controlada e financiada pelo governo ou instituições acadêmicas, mas a decisão de 
permitir o uso comercial abriu caminho para o desenvolvimento do comércio eletrónico.
\end{enumerate}

Istou levou ao levantamento de questões em como deveriamos lidar com a populariação da Internet.
O artico sobre a Governança na Internet por Hardy em 1994 questiona-se se deviamos lidar com 
as leis, costumes e regras associadas a sua Governança de maneira diferente a como lidamos 
no mundo real e refere a criação de uma "\textit{multistakeholder collaboration}" para lidar 
com essas questões.


\subsection{A explosão dos anos 2000 e a ICANN}
A explosão da bolha da Internet em 2000 resultou na consolidação e reestruturação da Internet
e como podiamos visualiza-la como uma rede que connecta não só militares e académicos, mas 
que connecta o Mundo.

A formação da ICANN surgiu na ideia de criar uma instituição para a coordenação global dos dominios
e IP's da Internet.

A pesquisa e debates durante este periodo mudaram drasticamente para se seria ideal 
criar verdadeiras instituições governamentais para a Internet.

Em relação á ICANN houve uma enorme controversia, já que era uma corporação privada sem
fins-lucrativos pertencente aos Estados Unidos que estava a ganhar autoridade global 
dos nomes de dominio e IPs. Esta desconfiança foi resolvida através de contratos privados
feitos para lidar com questões de politica em termos de concorrencia no mercado
comercial de nomes de dominios e alocacao de enderecos IP.

\subsection{Da ICANN até aos dias de hoje}
De 2003 a 2009 a Governança da Internet tornou-se totalmente reconhecida como um dominio 
de Governança global e o tema expandiu além da ICANN mas foi na \textit{World Summit on the Information Society} (WSIS)
que mudou como viamos a pratica da Governança.

Consistiu em dois eventos patrocinados pela ONU para criar uma plataforma \textit{multistakeholder}
para abordar questões mundiais e expandiu o ponto da Governança da Internet o publico
da Web. 

Foi decidido então centralizar o serviço de DNS através da ICANN e foi implementado um modelo de
governança corporativa que envolve os setores privados e publicos como responsáveis pela
manutenção da Internet.

\subsection{Conclusão}

A evolução da governança da Internet ao longo das últimas décadas tem sido marcada por 
uma transformação significativa. Inicialmente, a Internet surgiu como uma plataforma 
para conectar acadêmicos e militares, mas rapidamente se expandiu para se tornar uma rede 
global, acessível a indivíduos e organizações de todo o mundo.

Em conclusão, a governança da Internet continua a ser um campo de estudo em constante 
evolução, refletindo a rápida transformação da própria Internet. É um tema de grande
 importância acadêmica e prática, pois influencia a forma como a Internet é utilizada e 
 regulamentada em todo o mundo.
\section{A Organização da Internet}
\label{sec:organization}
A Internet é constituida por um modelo complexo de multiplas camadas e elementos interconectados.
Uma visão geral da estrutura da Internet consiste em:

\subsection{Infraestrutura Fisica}
No coração da Internet está a sua estrutura fisica, composta por uma rede de cabos 
de fibra óptica, cabos submarinos, satélites e torres de comunicação sem fio.
Essa infraestrutura é repsonsável por transmitir os dados entre os diferentes pontos
da rede, formando o \textit{backbone} essencial da Internet.

\subsection{Backbone da Internet}
Consiste em uma rede de alta velocidade que interliga os principais pontos de troca 
de trafego em todo o mundo. Este é operado por empresas de telecomunicações e ISPs.

\subsection{Provedores de Serviços de Internet (ISPs)}
Os ISPs desempenham um papel fundamental na organização da Internet, fornecendo acesso à
rede para utilizadores e empresas. Estas conectam os dispositivos à Internet por meio de 
diferentes tecnologias.

\subsection{Protocologos}
No nivel mais fundamental da Internet estão uma série de protocolos de comunicação, como 
o TCP/IP, que padronizam a troca de dados entre os dispositivos conectados. Estes garantem
uma comunicação eficiente e confiável em toda a rede global.

\subsection{Endereços IP e DNS}
São essenciais para a identificação e localização de recursos na Internet. A ICANN desempenha 
um papel central na coordenação global desses recursos, garantindo que cada dispositivos
e serviço tenha um identificado único reconhecivel.

\subsection{Conclusão}
A organização da Internet é uma combinação complexa de infraestrutura física, 
protocolos de comunicação e entidades reguladoras. Essa estrutura descentralizada 
e distribuída permite que a Internet seja uma rede global resiliente, escalável 
e capaz de atender às crescentes demandas de uma sociedade em constante evolução. 
\section{A Internet em Portugal}
\label{sec:portugalcase}

Portugal connectou-se à Internet no Outuno de 1991, como resultado do projecto "Serviço IP 
da RCCN". Em uma altura em que os pioneiros portugueses na Internet ainda só estavam a 
começar a tomar contacto com a Internet, nos US já era possível trocar \textit{e-mails}
entre computadores através de ARPANET.

\subsection{Acesso da Internet aos portugueses}
Entre 1991 e 1995, para além das universidades e instituos de investigação portugueses 
apoiados pela FCCN, as instituições e individuos que tiveram acesso a serviços da Internet
fizeram-no via a PUUG (Portuguese Unix Users Group), ou pela INESC (Instituto de Engenharia
de Sistemas e Computadores) sendo sócios dessas associações sem fins lucrativos.

\subsection{Modernização dos portugueses na Internet}
O dominio DNS ".pt" foi definido em Setembro de 1991 pelo NIC (Network Information Center)
reconheceu a sua existência, posicionando Portugal no mapa da Internet Global, já que  
os servidores sob nosso controlo seriam vistos na Internet com o nome “.pt”.

Por volta dos anos 2000, o panorama do acesso à Internet ja estava totalmente mudado com a 
maioria dos acessos providenciada de forma pública por operadores de telecomunicações
nacionais e internacionais. Esse aumento do acesso público à Internet democratizou o 
conhecimento e abriu novas oportunidades para os portugueses.

\subsection{Dificuldades dos Portugueses}
Apesar dos avanços, os portugueses enfrentam desafios significativos na era digital, 
incluindo questões de segurança cibernética, privacidade dos dados e exclusão digital. 

\subsection{Conclusão}
A modernização dos portugueses na Internet representa uma jornada contínua de 
transformação e adaptação à era digital. À medida que Portugal avança rumo a um 
futuro digital, é crucial enfrentar os desafios e aproveitar as oportunidades 
oferecidas pela Internet, garantindo que todos os cidadãos possam colher os benefícios 
de uma sociedade digital inclusiva e próspera.

\section{O futuro da Internet}
À medida que avançamos para o futuro, a Internet continua a desempenhar um papel 
central em nossas vidas, impulsionando a inovação, conectando pessoas e transformando 
a sociedade. 

No entanto, à medida que exploramos as possibilidades do amanhã, enfrentamos uma série 
de desafios e questões complexas que moldarão o curso da Internet ao longo do seu tempo 
de vida.

\subsection{Desafios de Segurança e Privacidade}
À medida que a Internet se torna mais onipresente e integrada em nossas vidas,
vimos a enfrentar desafios crescentes relacionados à segurança cibernética e à 
privacidade dos dados. O aumento crescente de dispositivos conectados, 
da sofisticação de ciberataques e da coleta de dados pessoais levantam preocupações
sobre como proteger as nossas informações.

\subsection{Computação Quântica}
Um dos principais objetivos da humanidade é a realização da computação quântica,
uma tecnologia revolucionária que promete transformar como vemos a Internet e como 
processamos informações. 

Esta tem capacidade de processamento exponencialmente superiores às dos computadores clássicos
e tem o potencial de resolver problemas complexos e realizar cálculos que estão além do alcance 
da computação tradicional.

\subsection{Guerra da Inteligência Artificial}
Enquanto exploramos as fronteiras das Inteligências artificais, nos deperamos com 
uma nova competição e colaboração entre sistemas de IA.

Esta guerra, tanto no campo militar quanto comercial, levanta questões éticas, legais e de 
segurança sobre o uso e o desenvolvimento de IA's como por exemplo o controle das máquinas
sobre os humanos. 

\subsection{Conclusão}
À medida que nos aventuramos no futuro da Internet, é imperativo abordar esses desafios 
de forma inteligente e colaborativa. Ao enfrentar questões como computação quântica, 
guerra de inteligência artificial e segurança cibernética podemos moldar um futuro onde a 
Internet continue a ser uma força positiva para a humanidade, capacitando-nos a alcançar 
novos patamares de progresso, inovação e conectividade global. 

Estes são pontos criticos de debate e reflexão à medida que avançamos para um 
mundo cada vez mais auotmatizado e interconectado.
\section{Conclusão}
À medida que refletimos sobre o passado, examinamos o presente e imaginos o futuro 
da Internet, vemos uma historia repleta de desafios e promessas. 

Desde os primórdios da ARPANET, impulsionada pela necessidade de comunicação em tempos de 
incerteza global durante a Guerra Fria, até a globalização da Internet das nossas vidas 
é de certeza uma transformação extraordinária.

No entanto, à medida que avançamos para o futuro, enfrentamos desafios complexos que 
exigem soluções inovadoras e colaborativas. A segurança cibernética e a privacidade dos 
dados são questões prementes que exigem atenção e ação imediata.
Além disso, os avanços na computação quântica e a crescente competição na guerra de 
inteligência artificial trazem novas oportunidades e dilemas éticos que devem ser abordados 
de forma cuidadosa e responsável.

No centro de tudo isso, a governança da Internet desempenha um papel crucial na 
definição do futuro da rede global. Com múltiplas partes interessadas e áreas funcionais 
diversas, a governança da Internet é um campo em constante evolução, adaptando-se às 
necessidades e desafios do mundo digital em rápida transformação.

A Internet é algo que nos connecta sem nós termos noção disso, e para garantir que continua 
a ser uma força positiva para a humanidade, é imperativo permanecermos vigilantes 
e comprometidos com os valores fundamentais de uma Internet aberta, inclusiva e segura.

\nocite{mueller2020inventing}
\nocite{musiani2013network}
\nocite{lukasik2010arpanet}
\nocite{publicoPioneirosInternet}
\nocite{Tictank}

\bibliographystyle{plainnat}
\bibliography{references}
\end{document}