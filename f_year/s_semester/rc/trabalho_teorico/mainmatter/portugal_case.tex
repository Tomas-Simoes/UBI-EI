\section{A Internet em Portugal}
\label{sec:portugalcase}

Portugal connectou-se à Internet no Outuno de 1991, como resultado do projecto "Serviço IP 
da RCCN". Em uma altura em que os pioneiros portugueses na Internet ainda só estavam a 
começar a tomar contacto com a Internet, nos US já era possível trocar \textit{e-mails}
entre computadores através de ARPANET.

\subsection{Acesso da Internet aos portugueses}
Entre 1991 e 1995, para além das universidades e instituos de investigação portugueses 
apoiados pela FCCN, as instituições e individuos que tiveram acesso a serviços da Internet
fizeram-no via a PUUG (Portuguese Unix Users Group), ou pela INESC (Instituto de Engenharia
de Sistemas e Computadores) sendo sócios dessas associações sem fins lucrativos.

\subsection{Modernização dos portugueses na Internet}
O dominio DNS ".pt" foi definido em Setembro de 1991 pelo NIC (Network Information Center)
reconheceu a sua existência, posicionando Portugal no mapa da Internet Global, já que  
os servidores sob nosso controlo seriam vistos na Internet com o nome “.pt”.

Por volta dos anos 2000, o panorama do acesso à Internet ja estava totalmente mudado com a 
maioria dos acessos providenciada de forma pública por operadores de telecomunicações
nacionais e internacionais. Esse aumento do acesso público à Internet democratizou o 
conhecimento e abriu novas oportunidades para os portugueses.

\subsection{Dificuldades dos Portugueses}
Apesar dos avanços, os portugueses enfrentam desafios significativos na era digital, 
incluindo questões de segurança cibernética, privacidade dos dados e exclusão digital. 

\subsection{Conclusão}
A modernização dos portugueses na Internet representa uma jornada contínua de 
transformação e adaptação à era digital. À medida que Portugal avança rumo a um 
futuro digital, é crucial enfrentar os desafios e aproveitar as oportunidades 
oferecidas pela Internet, garantindo que todos os cidadãos possam colher os benefícios 
de uma sociedade digital inclusiva e próspera.
