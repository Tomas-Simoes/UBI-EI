\section{O que é a Governança da Internet?}
\label{sec:internetgovernance}
A Governança da Internet é definida como "decisões no \textit{design} e desenvolvimento 
de tecnologias que são necessárias para manter a Internet operacional com a implementação
de políticas novas em volta das mesmas".

Ao longo do tempo, pesquisadores assim como profissionais na área têm tentado expandir o 
conhecimento deste extenso dominio, apresentando divisões para os problemas em questão.

Uma dessas classificações, proposta por DeNardis e Raymond, segmenta a Governança da Internet em seis áreas funcionais:

\begin{enumerate}
  \item Controle de Recursos Críticos da Internet (CIRs);
  \item Estabelecimento de \textit{standards} da Internet;
  \item Coordenação de Acesso e Interligação;
  \item Governança de Cibersegurança;
  \item Intermediação de Informações;
  \item Direitos de Propriedade Intelectual baseados em Arquitetura;
\end{enumerate}

Estas áreas têm diferentes estratégias e modelos de admistração, cada um adaptado
para ser mais eficiente na sua área especifica. No entanto, todos os pontos 
convergem para um objetivo comum.

Examinando a trajetória da Internet ao longo do tempo, percebe-se que ela foi 
observada em diferentes contextos históricos e culturais, acompanhada de 
diversas expectativas e decepções.

Por conta disso, foi preciso mantar uma abordagem de flexivel face ao desenvolvimento
da arquitetura da Internet, permitindo que novos atores com diferentes ideias entrassem 
e modificassem a estrutura conforme desejado.




