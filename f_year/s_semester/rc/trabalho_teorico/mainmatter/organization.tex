\section{A Organização da Internet}
\label{sec:organization}
A Internet é constituida por um modelo complexo de multiplas camadas e elementos interconectados.
Uma visão geral da estrutura da Internet consiste em:

\subsection{Infraestrutura Fisica}
No coração da Internet está a sua estrutura fisica, composta por uma rede de cabos 
de fibra óptica, cabos submarinos, satélites e torres de comunicação sem fio.
Essa infraestrutura é repsonsável por transmitir os dados entre os diferentes pontos
da rede, formando o \textit{backbone} essencial da Internet.

\subsection{Backbone da Internet}
Consiste em uma rede de alta velocidade que interliga os principais pontos de troca 
de trafego em todo o mundo. Este é operado por empresas de telecomunicações e ISPs.

\subsection{Provedores de Serviços de Internet (ISPs)}
Os ISPs desempenham um papel fundamental na organização da Internet, fornecendo acesso à
rede para utilizadores e empresas. Estas conectam os dispositivos à Internet por meio de 
diferentes tecnologias.

\subsection{Protocologos}
No nivel mais fundamental da Internet estão uma série de protocolos de comunicação, como 
o TCP/IP, que padronizam a troca de dados entre os dispositivos conectados. Estes garantem
uma comunicação eficiente e confiável em toda a rede global.

\subsection{Endereços IP e DNS}
São essenciais para a identificação e localização de recursos na Internet. A ICANN desempenha 
um papel central na coordenação global desses recursos, garantindo que cada dispositivos
e serviço tenha um identificado único reconhecivel.

\subsection{Conclusão}
A organização da Internet é uma combinação complexa de infraestrutura física, 
protocolos de comunicação e entidades reguladoras. Essa estrutura descentralizada 
e distribuída permite que a Internet seja uma rede global resiliente, escalável 
e capaz de atender às crescentes demandas de uma sociedade em constante evolução. 