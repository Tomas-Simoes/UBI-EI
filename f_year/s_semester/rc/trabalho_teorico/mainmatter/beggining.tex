\section{O inicio da Internet}
\label{sec:beggining}

\subsection{A bomba atómica e a Internet}
O primeiro passo na criação da Rede de Redes foi feito em 29 de agosto de 1949
quando a União Soviética detonou a sua primeira arma nuclear e os Estados Unidos
começaram a trabalhar nas suas própias técnologias, como um estação de radares
contra força-area. 

No meio disso, varios acontecimentos da Guerra Fria aconteceram, que não vamos
entrar a fundo. No entanto, tudo fez com que alguns anos depois
a ARPANET surgiu como uma defesa aos ataques da União Soviética.

O \textit{US Department of Defense} (DoD) estava preocupada em como 
os seus lideres politicos e militares iam continuar conectados após
um ataque nuclear inicial por parte da União Soviética.

A solução técnica foi criar novas maneiras de \textit{routing} e \textit{switching}
de forma a terem sistemas de comunicação decentrealizados que poderiam sobreviver 
a qualquer ataque.

O sistema proposto se assemelhava notavelmente ao que a ARPANET eventualmente
se tornou.

Podemos assim dizer que a criação Internet teve uma relação bastante
intima com a criação da bomba atómica.

\subsection{Surgimento da ARPANET}
Criada pelo \textit{US Defense Advanced Reasearch Projects Agency} (ARPA), foi a primeira
rede de computadores experimental estabelicida, onde novos \textit{softwares} 
eram testados. Esta rede ligava 4 universidades, o que também permitia 
a divulgação de informação de forma muito mais eficaz.

Esta funcionava através de um sistema de transmissão de dados no qual as informações
são divididas em pequenos pacotes, que por sua vez contém trechos dos dados e
o endereço do destinatário. Trouxe outras estratégias inovadoras como 
o "\textit{layering}", que permitia que os componenetes da rede podessem
ser alterados de forma independente.

A ideia da rede com "\textit{open-architecture}" surgiu logo em 1972, por Kahn que 
definiu quatro regras:

\begin{enumerate}
  \item Cada rede deveria ser capaz de funcionar independentemente e não precisaria 
passar por mudanças internas para se conectar à Internet.
  \item A transmissão de dados seria feita com a melhor fidelidade possível. 
Se um pacote de dados não chegasse ao destino, ele seria rapidamente retransmitido a 
partir da origem.
  \item Dispositivos conhecidos hoje em dia como \textit{gateways} e \textit{rooters} seriam \
utilizados para conectar as redes. Esses dispositivos não manteriam informações 
sobre os pacotes que passam por eles, mantendo assim uma  estrutura simples e evitando adaptações complicadas.
\end{enumerate}

Durante a década de 70 também surgiram novos protocologos de comunicação, incluido o TCP/IP, que se
tornou o padrão para a ARPANET e as suas redes.

Ao longo do tempo, varias versões da ARPANET surgiram e a humanidade continou a avançar 
na ideia central de uma rede de comunicação.



