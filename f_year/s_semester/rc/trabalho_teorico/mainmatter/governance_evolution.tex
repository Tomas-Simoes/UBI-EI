\section{Evolução da Governança da Internet}
\label{sec:governanceevo}
O termo "Governança da Internet" não está em nenhum artigo escolar até 1995. No entanto começaram a surgir pesquisadores com propostas em como aplicar 
Governança e leis na Internet.

O surgimento desta área de pesquisa converge para acontecimentos em meados do século 19, 
em que a Internet começou a emergir como uma massiva mídia
causada por 3 principais eventos:

\begin{enumerate}
  \item O desenvolvimento e adoção do protocologo WWW (1989-1993);
  \item[] Introduzindo a ideia de páginas web acessíveis a todos e permitiu a criação 
de uma rede de informações interligadas que poderiam ser acessadas por meio de navegadores
da \textit{web}.
  
  \item O surgimento de provedores de serviços de Internet (ISPs);
  \item[] Levou à popularização do acesso à Internet em domicilios e empresas;

  \item A publicação de \textit{web browsers} gratuitos para utilização (1991);
  \item[]Os \textit{web browsers} geralmente eram comercializados e a disponibilidade 
dos mesmos facilitou o acesso à Internet e abriu portas para novos individuos na sua
população.

  \item Privatização da Internet e a sua abertura para uso comercial 
pela \textit{US National Science Foundation} (1995); 
  \item[]Antigamente, a infraestrutura principal da Internet era predominanmente 
controlada e financiada pelo governo ou instituições acadêmicas, mas a decisão de 
permitir o uso comercial abriu caminho para o desenvolvimento do comércio eletrónico.
\end{enumerate}

Istou levou ao levantamento de questões em como deveriamos lidar com a populariação da Internet.
O artico sobre a Governança na Internet por Hardy em 1994 questiona-se se deviamos lidar com 
as leis, costumes e regras associadas a sua Governança de maneira diferente a como lidamos 
no mundo real e refere a criação de uma "\textit{multistakeholder collaboration}" para lidar 
com essas questões.


\subsection{A explosão dos anos 2000 e a ICANN}
A explosão da bolha da Internet em 2000 resultou na consolidação e reestruturação da Internet
e como podiamos visualiza-la como uma rede que connecta não só militares e académicos, mas 
que connecta o Mundo.

A formação da ICANN surgiu na ideia de criar uma instituição para a coordenação global dos dominios
e IP's da Internet.

A pesquisa e debates durante este periodo mudaram drasticamente para se seria ideal 
criar verdadeiras instituições governamentais para a Internet.

Em relação á ICANN houve uma enorme controversia, já que era uma corporação privada sem
fins-lucrativos pertencente aos Estados Unidos que estava a ganhar autoridade global 
dos nomes de dominio e IPs. Esta desconfiança foi resolvida através de contratos privados
feitos para lidar com questões de politica em termos de concorrencia no mercado
comercial de nomes de dominios e alocacao de enderecos IP.

\subsection{Da ICANN até aos dias de hoje}
De 2003 a 2009 a Governança da Internet tornou-se totalmente reconhecida como um dominio 
de Governança global e o tema expandiu além da ICANN mas foi na \textit{World Summit on the Information Society} (WSIS)
que mudou como viamos a pratica da Governança.

Consistiu em dois eventos patrocinados pela ONU para criar uma plataforma \textit{multistakeholder}
para abordar questões mundiais e expandiu o ponto da Governança da Internet o publico
da Web. 

Foi decidido então centralizar o serviço de DNS através da ICANN e foi implementado um modelo de
governança corporativa que envolve os setores privados e publicos como responsáveis pela
manutenção da Internet.

\subsection{Conclusão}

A evolução da governança da Internet ao longo das últimas décadas tem sido marcada por 
uma transformação significativa. Inicialmente, a Internet surgiu como uma plataforma 
para conectar acadêmicos e militares, mas rapidamente se expandiu para se tornar uma rede 
global, acessível a indivíduos e organizações de todo o mundo.

Em conclusão, a governança da Internet continua a ser um campo de estudo em constante 
evolução, refletindo a rápida transformação da própria Internet. É um tema de grande
 importância acadêmica e prática, pois influencia a forma como a Internet é utilizada e 
 regulamentada em todo o mundo.