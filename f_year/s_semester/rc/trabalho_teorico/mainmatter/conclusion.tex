\section{Conclusão}
À medida que refletimos sobre o passado, examinamos o presente e imaginos o futuro 
da Internet, vemos uma historia repleta de desafios e promessas. 

Desde os primórdios da ARPANET, impulsionada pela necessidade de comunicação em tempos de 
incerteza global durante a Guerra Fria, até a globalização da Internet das nossas vidas 
é de certeza uma transformação extraordinária.

No entanto, à medida que avançamos para o futuro, enfrentamos desafios complexos que 
exigem soluções inovadoras e colaborativas. A segurança cibernética e a privacidade dos 
dados são questões prementes que exigem atenção e ação imediata.
Além disso, os avanços na computação quântica e a crescente competição na guerra de 
inteligência artificial trazem novas oportunidades e dilemas éticos que devem ser abordados 
de forma cuidadosa e responsável.

No centro de tudo isso, a governança da Internet desempenha um papel crucial na 
definição do futuro da rede global. Com múltiplas partes interessadas e áreas funcionais 
diversas, a governança da Internet é um campo em constante evolução, adaptando-se às 
necessidades e desafios do mundo digital em rápida transformação.

A Internet é algo que nos connecta sem nós termos noção disso, e para garantir que continua 
a ser uma força positiva para a humanidade, é imperativo permanecermos vigilantes 
e comprometidos com os valores fundamentais de uma Internet aberta, inclusiva e segura.