\chapter{Evolução da Governança da Internet}
% OU \chapter{Trabalhos Relacionados}
% OU \chapter{Engenharia de Software}
% OU \chapter{Tecnologias e Ferramentas Utilizadas}
\label{chap:evolução-da-governança-da-internet}

\section{Introdução}
\label{chap2:sec:intro}
Cada capítulo \underline{intermédio} deve começar com uma breve introdução onde é explicado com um pouco mais de detalhe qual é o tema deste capítulo, e como é que se encontra organizado (i.e., o que é que cada secção seguinte discute). 

%\section{Citações e Referências Cruzadas -- [RETIRAR DA VERSÃO FINAL]}
%\label{chap2:sec:citacoes}

Para se referenciarem outras secções, usar \texttt{\textbackslash{}ref\{label\}}, e.g., para citar a secção da Introdução deste capítulo, usar \texttt{\textbackslash{}ref\{chap2:sec:intro\}}. O resultado é: a secção~\ref{chap2:sec:intro} contém a introdução deste capítulo.

Para se citarem fontes bibliográficas, \underline{colocar a entrada certa} no ficheiro \texttt{bibiografia.bib} e usar o comando \texttt{\textbackslash{}cite\{label-da-referencia\}}, ligando o comando com a palavra que o antecede com um til. Por exemplo, para citar a referência eletrónica \emph{The Not So Short Introduction to \LaTeX{}}~\cite{short}, deve incluir-se o trecho seguinte no ficheiro \texttt{bibiografia.bib} e usar \texttt{\textbackslash{}cite\{\underline{short}\}} para a citação (citação incluída nesta mesma frase):
%
\small
\begin{verbatim}
@MISC{short,
  author = {Tobias Oetiker and Hubert Partl and 
            Irene Hyna and Elisabeth Schlegl},
  title  = "{The Not So Short Introduction to \LaTeX{}}",
  year   = 2018,
  note   = {[Online] \url{https://tobi.oetiker.ch/lshort/lshort.pdf}. 
            Último acesso a 12 de Março de 2019}
}
\end{verbatim}
\normalsize


\section{Secções Intermédias}
\label{chap2:sec:...}

\section{Conclusões}
\label{chap2:sec:concs}
Cada capítulo \underline{intermédio} deve referir o que demais importante se conclui desta parte do trabalho, de modo a fornecer a motivação para o capítulo ou passos seguintes.